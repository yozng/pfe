\documentclass[a4paper,12pt,twoside]{report} %beamer : presentation
%\usepackage[utf8]{inputenc}  % Encodage utf8 à conserver ! %latin1
\usepackage[T1]{fontenc}
\usepackage[french]{babel}   % Pour adopter les règles de typographie française
%\usepackage[left=cm,right=cm,top=cm,bottom=cm]{geometry}
% Les bibliothèques de l'American Mathematical Society pour les maths
\usepackage{amsmath}
\usepackage{amsfonts}
\usepackage{amssymb}
\usepackage{amsthm}

\usepackage{lettrine}

\usepackage{xcolor}
%%%%%%%%%%%%%%%%%%%%%%%%%%%%%%%%%%%%%%%%%%%%%%
%\usepackage{fancyhdr}
%\pagestyle{fancy}
%\fancyhead[CE]{\leftmark}
%\fancyhead[CO]{\rightmark}
%\fancyfoot[CE,CO]{\thepage}
%\fancyfoot[RE]{IREM de Lyon}
%\fancyfoot[LO]{\LaTeX{}\ldots{}pour le prof de maths}
%%%%%%%%
%\renewcommand{\headrulewidth}{0.4pt}
%%%%%%%%%%%%%%%%%%%%%%%%%%%%%%%%%%%%%%%%%%%%%%


 \numberwithin{equation}{section} %% Comment out for sequentially-numbered
 \numberwithin{figure}{section} %% Comment out for sequentially-numbered

% Les définitions, théorèmes et autres lemmes, corollaires, propositions,  
 \newtheorem{theo}{Th\'eor\`eme}[section]
 \newtheorem{defi}[theo]{D\'efinition}
 \newtheorem{prop}[theo]{Proposition} %%Delete [theo] to re-start numbering
 \newtheorem{exem}[theo]{Exemple}
 \newtheorem{exer}[section]{Exercise}%%Delete [section] for sequential numbering
 \newtheorem{rema}[theo]{Remarque}   
 \newtheorem{lemm}[theo]{Lemme} %%Delete [theo] to re-start numbering   
 \newtheorem{coro}[theo]{Corollaire} %%Delete [theo] to re-start numbering  
 \newcommand{\HRule}{\rule{\linewidth}{0.5mm}}

\usepackage{fullpage} % Agrandit les dimensions du texte (hauteur, largeur,
                      % etc.) par rapport à celles par défaut. Attention
                      % ce package ne se trouve pas dans toutes les
                      % distributions LaTeX

\usepackage{graphicx}
%\topmargin=-.5in \textheight=9.in\leftmargin=-.2in
\begin{document} 
\begin{titlepage}
\begin{center}
\noindent {\large \textbf{UNIVERSITÉ MOHAMMED V DE RABAT}} \\
\vspace*{0.2cm}
\noindent {\Huge \textbf{Faculté des Sciences }} \\
\vspace*{0.5cm}
\includegraphics[scale=.29]{LogoFsr.png}
\vspace*{1cm}




\noindent \LARGE \textbf{Département d'Informatique}
\vspace*{.5cm}



\noindent \Large \textbf{Filière Licence Fondamentale \\ en Sciences Mathématiques et Informatique} 
\vspace*{1cm}


\HRule \\[0.1cm]
    { \huge \bfseries PROJET DE FIN D'ÉTUDES \\[0.1cm] }
    \HRule \\[0.2cm]


\vspace*{0.4cm}
\noindent \large {Intitulé :}

\vspace*{0.8cm}
\noindent {\Large \textbf{Digitalisation et Analyse des Données du Parcours Doctoral}} \\
\vspace*{0.3cm}
\noindent \Large Présenté par: \\  \textsc{\textbf{Hiba HAMRITI} \item \textbf{Omaima ALMA}} \\
\vspace*{0.2cm}


\vspace*{0.2cm}
\noindent \large soutenu le 17 Juin 2025 devant le Jury\\
\vspace*{0.5cm}
\end{center}


\noindent \large 
\begin{tabular}{lll}


Pr. Prénom Nom   &  Faculté des Sciences - Rabat & \textit{Président}	\\
Pr. Soumia ZITI   &  Faculté des Sciences - Rabat & \textit{Encadrante}	\\
Pr. Prénom Nom &     Faculté des Sciences - Rabat  & \textit{Examinateur} 

\end{tabular}

\vfill
\begin{center}
Année Universitaire 2024-2025
\end{center}
\end{titlepage}
\sloppy

\titlepage


\pagenumbering{roman}
 \cleardoublepage 
 
\chapter*{Remerciements}
\addcontentsline{toc}{chapter}{Remerciements}



Au terme de ce travail, nous tenons à exprimer notre profonde gratitude et nos sincères remerciements à toutes les personnes qui ont contribué, de près ou de loin, à la réalisation de ce projet de fin d'études.\\

Un merci tout particulier à notre encadrante, Madame \textbf{Soumia ZITI}, pour sa patience, sa confiance en nous. Elle a su nous orienter avec justesse sans jamais nous imposer une voie, nous laissant l’espace nécessaire pour réfléchir, expérimenter et apprendre par nous-mêmes. Son accompagnement, nous a encouragées à développer notre autonomie et à croire en nos capacités tout au long de ce projet.\\

Nous remercions également les \textbf{membres du jury} pour le temps et l’attention qu’ils ont accordés à notre travail. Leurs retours, remarques et suggestions nous sont précieux et nous aideront à continuer de progresser.\\

Nous n’oublions pas toutes les personnes qui nous ont soutenues, encouragées ou simplement écoutées pendant les moments de doute ou de fatigue. À nos \textbf{familles}, nos \textbf{amis}, nos \textbf{camarades} et nos \textbf{proches}, un grand merci pour votre présence et votre soutien, parfois discret mais toujours précieux.\\

Nous tenons également à nous féliciter mutuellement pour notre persévérance, notre engagement et l’esprit de collaboration dont nous avons fait preuve tout au long de cette expérience.\\

Ce projet représente bien plus qu'une étape académique ; il symbolise le fruit d'un travail d'équipe, l'acquisition de compétences précieuses et les rencontres inspirantes qui ont enrichi notre formation.\\




\chapter*{Résumé}

\addcontentsline{toc}{chapter}{Résumé}



Un résumé en français et un résumé en anglais sont obligatoires dans un mémoire de fin d'études.\\

Ils peuvent être mis dans une même page comme le cas ci-dessous ou dans deux pages séparés.\\

Dans le résumé, vous devez donner d'une façon concise et précise le résumé de votre travail durant les 3 mois de réalisation de votre Projet de fin d'étdudes.\\

Le résumé devra être suivi d'une liste de 5 mots clés maximum en relation avec le sujet du PFE.
\\
\\  
{\large\textbf{Mots clés :}}
Analyse de donnée, LaTeX, PFE, SMI, FSR

\chapter*{Abstract}

\addcontentsline{toc}{chapter}{Abstract}



The main objective of this ...
\\
\\
{\large\textbf{Keywords:}}
TeX, LaTeX, PFE, SMI, FSR


\tableofcontents
%\mainmatter
%\pagestyle{headings}

\chapter*{Introduction}
\addcontentsline{toc}{chapter}{Introduction}

Le document suivant représente un modèle à suivre lors de la rédaction de votre mémoire de Projet de Fin d'études et en même temps un petit guide de PFE.\\

Le nombre de chapitres ainsi que le nombre de sections et sous-sections au niveau de chaque chapitre dépend du contenu de chaque sujet.\\

Au niveau du chapitre 2, vous trouverez certaines consignes à suivre avant et après votre soutenance.







\pagenumbering{arabic}
\chapter{Module PFE}

Le PFE ou Projet de Fin d'études est l'équivalent de deux modules Projets Tutorés: PT1 et PT2.\\

Le projet tutoré a pour objectif d'initier les étudiants à la recherche scientifique expérimentale, à la recherche bibliographique, aux stages en entreprises ainsi qu'à la rédaction de rapports scientifiques. \\ 

Votre mémoire de fin d'études devra être rédigé en LaTex.\\


 
 
\section{Première section}  
\subsection{Première sous-section}

\subsection{Deuxième sous-section}
On peut insérer une figure en LaTex comme suit:
\begin{figure}
	\begin{center}
	\includegraphics[width=12cm,height=5cm]{fig1.png}
	\caption{Question}
	\end{center}
\end{figure}



\section{Deuxième section}

\subsection{Première sous-section}

\subsection{Deuxième sous-section}


 



 





\chapter{Avant et après la soutenance de PFE}

Les soutenances de PFE peuvent être effectuées durant les semaines du 06 et du 13 juin 2016, qui correspondent aux semaines de Délibérations et de Rattrapage.

\section{Préparation à la soutenance} 
Ci-dessous certains points à respecter afin de bien préparer votre soutenance:\\

\begin{itemize}
	\item Le mémoire de PFE devra être rédigé en LaTeX en respectant le modèle adopté par le département d'informatique.\\
	
	\item Avant la soutenance, l?encadrant désignera les membres du Jury qui vont évaluer le travail du binôme.
	
	\item Chaque binôme devra remettre à son encadrant ainsi qu'aux  membres du jury une copie du mémoire de PFE (rassemblée uniquement avec reliure), au moins 5 jours avant la date de la soutenance.
	
	\item	Le jour de la soutenance, chaque binôme devra ramener au président du Jury un PV de soutenance en y indiquant les noms et prénoms de chaque étudiant ainsi que le titre du mémoire de PFE (le PV est disponible au secrétariat du département d?informatique).
	
	\item	Les soutenances sont publiques.
\end{itemize}



\section{Après la soutenance}
 \begin{itemize}
 	\item Les versions définitives des mémoires sont à réaliser après soutenance en y incluant les membres de jury et toutes leurs remarques. \\
 	
 	\item La version  définitive de chaque mémoire  doit être rassemblée avec "colle" et non avec reliure, et mettre au niveau de la dernière page de couverture, un résumé en français et anglais.\\
 	 		
 	\item Il faudra remettre une copie de la version finale du mémoire ainsi qu'un CD-ROM contenant le mémoire de PFE, la présentation et éventuellement l'application réalisée, à :\\
 	\begin{itemize}
 		\item chaque membre du jury de la soutenance 
 		\item la bibliothèque de la Faculté des Sciences de Rabat.
  	\end{itemize}
  	
  	\item L'application sujette du Projet de fin d?études pourra éventuellement être publiée sur le site du département informatique si les membres du jury de la soutenance le recommandent dans le PV de soutenance.\\
  	
  	\item Toute application doit être en format code source et binaire et accompagnée d'un fichier d'installation appelé Readme.\\
  \end{itemize}
 	

\chapter*{Conclusion}
\addcontentsline{toc}{chapter}{Conclusion}

Ce dernier chapitre concerne la conclusion générale de votre projet.

\begin{thebibliography}{10} % 0 si vous avez moins de 9 références sinon 10

\bibitem{1} 
Andrew Tanenbaum (Université libre d?Amsterdam), « Systèmes d?exploitation », 3ème édition, Nouveaux Horizons, 2008.
\bibitem{2}
http://www.linux.org/


\end{thebibliography}




\end{document}
