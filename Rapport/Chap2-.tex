\chapter{Avant et après la soutenance de PFE}

Les soutenances de PFE peuvent être effectuées durant les semaines du 06 et du 13 juin 2016, qui correspondent aux semaines de Délibérations et de Rattrapage.

\section{Préparation à la soutenance} 
Ci-dessous certains points à respecter afin de bien préparer votre soutenance:\\

\begin{itemize}
	\item Le mémoire de PFE devra être rédigé en LaTeX en respectant le modèle adopté par le département d'informatique.\\
	
	\item Avant la soutenance, l?encadrant désignera les membres du Jury qui vont évaluer le travail du binôme.
	
	\item Chaque binôme devra remettre à son encadrant ainsi qu'aux  membres du jury une copie du mémoire de PFE (rassemblée uniquement avec reliure), au moins 5 jours avant la date de la soutenance.
	
	\item	Le jour de la soutenance, chaque binôme devra ramener au président du Jury un PV de soutenance en y indiquant les noms et prénoms de chaque étudiant ainsi que le titre du mémoire de PFE (le PV est disponible au secrétariat du département d?informatique).
	
	\item	Les soutenances sont publiques.
\end{itemize}



\section{Après la soutenance}
 \begin{itemize}
 	\item Les versions définitives des mémoires sont à réaliser après soutenance en y incluant les membres de jury et toutes leurs remarques. \\
 	
 	\item La version  définitive de chaque mémoire  doit être rassemblée avec "colle" et non avec reliure, et mettre au niveau de la dernière page de couverture, un résumé en français et anglais.\\
 	 		
 	\item Il faudra remettre une copie de la version finale du mémoire ainsi qu'un CD-ROM contenant le mémoire de PFE, la présentation et éventuellement l'application réalisée, à :\\
 	\begin{itemize}
 		\item chaque membre du jury de la soutenance 
 		\item la bibliothèque de la Faculté des Sciences de Rabat.
  	\end{itemize}
  	
  	\item L'application sujette du Projet de fin d?études pourra éventuellement être publiée sur le site du département informatique si les membres du jury de la soutenance le recommandent dans le PV de soutenance.\\
  	
  	\item Toute application doit être en format code source et binaire et accompagnée d'un fichier d'installation appelé Readme.\\
  \end{itemize}
 	